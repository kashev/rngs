The applications of randomness are far reaching. From statistics to simulation, there is a large need for random number generators to perform quickly, generate seemingly random numbers, yet be reproducible in case something needs to be random yet reproducible. Thus, the creation of pseudo random number generators, or PRNGs, has become a large research area in modeling and computer simulation. In the field of analysis of computing systems, simulation is an extremely important part of the modeling process. Today's stochastic models become so complex that solving these models analytically becomes impossible. Simulations driven by randomness are common, and motivate the study in this paper.

\subsection{Goals of the Project}

We plan to create implementations of several random number generators, and then analyze their effectiveness via several different metrics. Our plan for implementing the PRNGs is in Section~\ref{sec:software_plan}, and our plan for the analysis of these generators is in Section~\ref{sec:analysis_plan}. We aim to create usable PRNGs, and then prove that they are usable as engines for simulations by passing common statistical tests.
