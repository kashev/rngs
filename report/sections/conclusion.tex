Hopefully at this point, one is convinced not only that testing random number generators is a tricky art which is difficult to get right, but that despite this, there are many reasonable choices for RNGs which pass the tests which we have. So, what is a user of Random Numbers to do?!

It turns out that the choice is more or less made for most users. By far, the most common random number in service today is the Mersenne Twister. It is the default RNG in Python, MATLAB, the GNU Scientific Library, as well as being available and encouraged in C++11. The trade offs for memory and speed efficiency simply don't matter as much to most users, especially with modern computers. Thus, a RNG with a large period like the Mersenne Twister, which is also proved to be well distributed in higher dimensions, seems like an obvious choice.

Not everyone agrees with these programming language designers! George Marsaglia, a noted RNG researcher, developed equally performant on test suites like Dieharder, generators of the CMWC variety which are faster, store less state, and have equally long periods \cite{Marsaglia:2003:SRN:769800.769827}. Additionally, the developers of the Mersenne Twister have created a different class of generators called WELL, which displays similar desirable properties \cite{Panneton:2006:ILG:1132973.1132974}. These same researchers have also created different versions of the Mersenne Twister, which take greater advantage of modern CPU architecture to gain higher performance \cite{sfmt}. Still other RNGs designed specifically for GPUs are gaining popularity in some applications \cite{Passerat-Palmbach:2011:PNG:2060104.2060643,Zafar:2010:GRN:1921479.1921500}. These generators are on the cutting edge of random number generation research, and as such, it will take longer for programming language designers to adopt them.

Ultimately, it is the job of the system designer to carefully weigh the trade offs of different random number generation techniques and chose one for their system. Luckily, they will have tools like Dieharder to aid them in their decision.
