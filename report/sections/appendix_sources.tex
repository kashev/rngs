This section includes some source files from the code base. Interested readers are encouraged to view the Github Repository for all code \cite{github_repo}.

The first file included is \texttt{rng.h} in Section~\ref{sec:rng_h}. This header file defines the interfaces for all the implemented RNGs. The next included sources are \texttt{xorshift.h} and \texttt{xorshift.cpp}, in Sections~\ref{sec:xorshift_h} and~\ref{sec:xorshift_cpp}, respectively. This should give the reader a feel for what the internals of the generators look like. Finally, the Python calling script, \texttt{rngs.py} in Section~\ref{sec:rngs_py}, which demonstrates hooking up the produced RNG binary to Dieharder, is included as well.

\subsection{rng.h}
\label{sec:rng_h}
The source for \texttt{rng.h}, the basic skeleton of all implemented generators.
\lstinputlisting[language=C++,
                 basicstyle=\ttfamily\footnotesize]
    {../code/rng.h}

\newpage
\subsection{xorshift.h}
\label{sec:xorshift_h}
The source for \texttt{xorshift.h}, an example generator implementation.
\lstinputlisting[language=C++,
                 basicstyle=\ttfamily\footnotesize]
    {../code/xorshift.h}

\newpage
\subsection{xorshift.cpp}
\label{sec:xorshift_cpp}
The source for \texttt{xorshift.cpp}, an example generator implementation.
\lstinputlisting[language=C++,
                 basicstyle=\ttfamily\footnotesize]
    {../code/xorshift.cpp}

\newpage
\subsection{rngs.py}
\label{sec:rngs_py}
The source for \texttt{rngs.py}, the main code for the RNG binary.
\lstinputlisting[language=Python,
                 basicstyle=\ttfamily\footnotesize]
    {../code/rngs.py}

